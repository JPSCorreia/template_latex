\section{Conclusão} \label{section: introdução}
Uma análise detalhada dos sistemas operativos \textit{open source} e proprietários revela uma série de considerações importantes para empresas que queiram adotá-los.
\par \vspace{6pt}
A importância do \textit{kernel} como o núcleo de qualquer sistema operativo é inegável, influenciando diretamente o desempenho, a estabilidade e a segurança do sistema. O \textit{kernel} Linux, em particular, destaca-se como um exemplo proeminente, oferecendo uma vasta gama de utilizações em diferentes contextos.
\par \vspace{6pt}
O licenciamento \textit{open source} apresenta tanto oportunidades quanto desafios, oferecendo liberdade e flexibilidade aos utilizadores, mas exigindo uma compreensão clara das obrigações legais associadas. No que diz respeito à segurança, embora os sistemas operativos \textit{open source} geralmente desfrutem de uma reputação de segurança robusta, é crucial adotar práticas de segurança adequadas e estar ciente das vulnerabilidades potenciais.
\par \vspace{6pt}
Para as empresas, a adoção de software \textit{open source} em ambientes empresariais pode oferecer uma série de vantagens, incluindo custos reduzidos, maior flexibilidade e acesso a uma vasta comunidade de desenvolvedores em troca de menos suporte profissional e estabilidade.
\par \vspace{6pt}
Em última análise, a escolha entre sistemas operativos \textit{open source} e proprietários é uma decisão estratégica que deve ser cuidadosamente ponderada, levando em consideração as necessidades específicas e os objetivos de negócios de cada empresa.
\par \vspace{6pt}
Ao compreender os diferentes aspectos envolvidos e considerando as implicações a longo prazo, as empresas devem fazer um estudo prévio, fazer um balanço entre os pontos positivos e negativos e tomar uma decisão informada sobre o tipo de sistemas a usar.
\par\vspace{24pt}

Todos os elementos do grupo participaram e alteraram o trabalho inteiro, mas o levantamento de informação de cada capítulo foi feito da seguinte maneira:
\begin{itemize}
\item \textbf{Daniel Quaresma}: Introdução e conclusão.
\item \textbf{Vladimiro Bonaparte}: Capítulo 2 e 3.
\item \textbf{Lucas Silvestre}: Capítulo 4 e 5.
\item \textbf{João Correia}: Capítulo 6 e 7.
\end{itemize}
