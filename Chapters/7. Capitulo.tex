\section{Software \textit{open source} em ambientes empresários} \label{section: aquisição}
\subsection{Porquê?}
Software de \textit{open source} ou proprietário? As empresas enfrentam esta questão fundamental quando se trata de escolher aplicações críticas para o seu negócio.
\par \vspace{6pt}
A decisão sobre o software adequado tem implicações a longo prazo e, portanto, deve ser considerada cuidadosamente. As empresas continuam a aumentar o seu uso de software de \textit{open source} nos últimos anos. Mesmo que muitas empresas não estejam cientes, o software de \textit{open source} já está presente em mais de 90\% das empresas em apoio à sua infraestrutura de TI. \cite{whyOpenSource}

\begin{figure}[H]
  \centering
  % width=\textwidth para imagem da largura do texto
  \includegraphics[scale=0.65]{Figures/0. General/enterprise_open_source.jpg}
  \caption{Empresas cada vez aderem mais a \textit{open source}}
  \label{Empresas cada vez aderem mais a open source}
\end{figure}

\subsection{Benefícios}
Algumas das principais razões que as empresas dão para a utilização de software de \textit{open source}: \cite{stateOfEnterpriseOpenSource}

\begin{itemize}

\item \textbf{Familiaridade}\
A compreensão por parte dos desenvolvedores da colaboração com comunidades, contribuições para projetos, entendimento de licenças e gestão de dependências.

\item \textbf{Apoio a comunidades}\
O sucesso do software \textit{open source} depende de comunidades saudáveis. Empresas podem contribuir com recursos, financiamento e feedback, fortalecendo tanto os projetos quanto a comunidade de \textit{open source}.

\item \textbf{Capacidade de influenciar o desenvolvimento dos recursos necessários}\
Software \textit{open source} oferece transparência e adaptação às necessidades específicas. As empresas podem influenciar ativamente o desenvolvimento de recursos, seja contribuindo com código ou fornecendo feedback valioso.

\item \textbf{Eficiência ao lidar com desafios técnicos}\
Acesso a uma vasta comunidade de especialistas permite solucionar desafios técnicos de forma rápida e eficaz. Fóruns online e grupos de discussão facilitam a obtenção de suporte.
\end{itemize}