\section{Licenciamento \textit{open source}} \label{section: licenciamento}
\subsection{Importância do licenciamento}
O licenciamento \textit{open source} é um aspeto crucial no desenvolvimento de software, servindo como uma fundação para a inovação e colaboração tecnológica.
As licenças \textit{open source} são rigorosamente aprovadas pela \textbf{\textit{Open Source Initiative}(OSI)} e asseguram que qualquer software sob estas licenças possa ser livremente utilizado, modificado e redistribuído. \cite{openSourceLicense}
\par \vspace{6pt}
Este sistema de licenciamento é vital para manter a integridade e a filosofia da partilha e colaboração que são centrais para a comunidade \textit{open source}.

\subsection{Definição e aprovação de licenças}
Segundo a \textbf{OSI}, uma licença só é considerada \textit{open source} se cumprir com a \textbf{\textit{Open Source Definition}(OSD)}. Esta definição inclui uma série de critérios projetados para proteger a liberdade do
utilizador e fomentar a inovação. Por exemplo, uma licença \textit{open source} deve permitir redistribuições livres do software, acesso ao código-fonte e criação de obras derivadas.
\par \vspace{6pt}
O processo de aprovação de uma licença pela \textbf{OSI} é um pilar essencial para garantir que estas normas sejam mantidas. Através de um processo de revisão pública, a comunidade \textit{open source} pode dar opiniões 
sobre novas licenças propostas para garantir que elas estejam alinhadas com os padrões estabelecidos. 
\par \vspace{6pt}
Este processo não apenas protege os direitos dos utilizadores e desenvolvedores, mas também mantém um 
padrão uniforme que facilita a colaboração e a partilha de tecnologia entre projetos e organizações.

\subsection{Tipos de licenças}
Existem dois tipos principais de licenças \textit{open source}: \textbf{\textit{copyleft}} e \textbf{permissivas}. As licenças \textbf{\textit{copyleft}}, como a \textbf{\textit{GNU General Public License}}, exigem que quaisquer versões
modificadas do software também sejam distribuídas com a mesma licença \textit{open source}. Isso garante que as liberdades concedidas pela licença original sejam mantidas em todas as versões derivadas do software. 
\par \vspace{6pt}
Por outro lado, as licenças \textbf{permissivas}, como a licença \textbf{MIT} e a licença \textbf{BSD}, são menos restritivas, permitindo que o software seja integrado em projetos proprietários. 
Essas licenças ainda garantem liberdades fundamentais, mas não exigem que as obras derivadas sejam distribuídas sob os mesmos termos \textit{open source}.

\subsection{Impacto do licenciamento}
O impacto do licenciamento \textit{open source} é profundo e abrangente. Este, permite que empresas, desde \textit{startups} até grandes organizações, inovem e construam sobre o trabalho existente sem as 
restrições de licenças de software proprietário.
\par \vspace{6pt}
Este ambiente de inovação aberta tem levado ao desenvolvimento de tecnologias significativas em campos como servidores \textbf{\textit{web}}, \textbf{\textit{smartphones}}, \textbf{automação empresarial}, \textbf{computação em \textit{cloud}} e a \textbf{economia partilhada}. 
O licenciamento \textit{open source} apoia a inovação contínua e a disseminação rápida de tecnologias emergentes, beneficiando tanto os desenvolvedores individuais quanto a indústria tecnológica em larga escala.
